\documentclass[draftclsnofoot,onecolumn,10pt,compsoc]{IEEEtran}
\usepackage[utf8]{inputenc}
\usepackage{color}
\usepackage{url}
\usepackage{hyperref}

\usepackage{graphicx} %package to manage images
\graphicspath{ {images/} }

\usepackage{enumitem}

\usepackage[letterpaper, margin=.75in]{geometry}

\newcommand{\toc}{\tableofcontents}

\usepackage{hyperref}
\usepackage{listings}

\definecolor{dkgreen}{rgb}{0,0.6,0}
\definecolor{gray}{rgb}{0.5,0.5,0.5}
\definecolor{mauve}{rgb}{0.58,0,0.82}

\renewcommand{\lstlistingname}{Code Example} % a listing caption title.
%\renewcommand{\lstlistlistingname}{List of \lstlistingname s} % list of lists -> list of Thread Program
\lstset{
    frame=single,
    language=C,
    columns=flexible,
    numbers=left,
    numbersep=5pt,
    numberstyle=\tiny\color{gray},
    keywordstyle=\color{blue},
    commentstyle=\color{dkgreen},
    stringstyle=\color{mauve},
    breaklines=true,
    breakatwhitespace=true,
    tabsize=4,
    captionpos=b
}

\def\name{Taylor Thomas}

%% The following metadata will show up in the PDF properties
\hypersetup{
  colorlinks = false,
  urlcolor = black,
  pdfauthor = {\name},
  pdfkeywords = {},
  pdftitle = {},
  pdfsubject = {},
  pdfpagemode = UseNone
}

\parindent = 0.0 in
\parskip = 0.1 in

\begin{document}

\begin{titlepage}
\title{Project 1}
\author{CS444 - Spring 2017 \\ Taylor Thomas, Aravind Parasurama, Justin Brown}
\maketitle
\begin{abstract}

\end{abstract}

\thispagestyle{empty} % gets rid of the "0" page number.

\end{titlepage}
%\newpage

\tableofcontents

\newpage

\section{Command Log}
\begin{enumerate}
	\item ssh os-class
	\item cd/scratch/spring2017/13-06
	\item git clone git://git.yoctoproject.org/linux-yocto-3.14
	\item cd linux-yocto-3.14
	\item git checkout v3.14.26
	\item source /scratch/opt/environment-setup-i586-poky-linux.csh
	\item cp /scratch/spring2017/files/config-3.14.26-yocto-qemu.config
	\item make menuconfig
	\item /LOCALVERSION
	\item 1
	\item edit value to: -13-06-hw1 
	\item make -j4 all
	\item cd ..
	\item gdb
	\item in terminal 2: 
	source /scratch/opt/environment-setup-i586-poky-linux.csh
	\item cp /scratch/spring2017/files/bzImage-qemux86.bin.
	\item /scratch/spring2017/files/core-image- lsb-sdk-qemux86.ext3 .
	\item qemu-system- i386 -gdb tcp::5601 -S -nographic -kernel bzImage-qemux86.bin -drive file=core-image- lsb-sdk- qemux86.ext3,if=virtio -enable-kvm -net none-usb-localtime--no-reboot -- append	"root=/dev/vda rw console=ttyS0 debug"
	\item in terminal 1: target remote :5601
	\item continue
	\item for username type: root
	\item uname -a
	\item reboot
	\item qemu-system- i386 -gdb tcp::5601 -S -nographic -kernel linux-yocto- 3.14/arch/x86/boot/bzImage  -drive file=core-image- lsb-sdk- qemux86.ext3,if=virtio -enable- kvm -net none -usb -localtime -- no-reboot -- append	"root=/dev/vda rw console=ttyS0 debug"
	\item target remote :5601
	\item for username type: root
	\item uname -a
	\item reboot
	\item q
	
\end{enumerate}
\section{Concurrency Solution}
The main point of the assignment is to create a program that prevents the
producer-consumer problem from occurring. In detail there are 
\section{Qemu Command-Line Flags}
\section{Version Control Log}
\section{Work Log}


\end{document}
