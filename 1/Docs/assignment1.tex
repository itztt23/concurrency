\documentclass[draftclsnofoot,onecolumn,10pt,compsoc]{IEEEtran}
\usepackage[utf8]{inputenc}
\usepackage{color}
\usepackage{url}
\usepackage{hyperref}

\usepackage{graphicx} %package to manage images
\graphicspath{ {images/} }

\usepackage{enumitem}

\usepackage[letterpaper, margin=.75in]{geometry}

\newcommand{\toc}{\tableofcontents}

\usepackage{hyperref}
\usepackage{listings}

\definecolor{dkgreen}{rgb}{0,0.6,0}
\definecolor{gray}{rgb}{0.5,0.5,0.5}
\definecolor{mauve}{rgb}{0.58,0,0.82}

\renewcommand{\lstlistingname}{Code Example} % a listing caption title.
%\renewcommand{\lstlistlistingname}{List of \lstlistingname s} % list of lists -> list of Thread Program
\lstset{
    frame=single,
    language=C,
    columns=flexible,
    numbers=left,
    numbersep=5pt,
    numberstyle=\tiny\color{gray},
    keywordstyle=\color{blue},
    commentstyle=\color{dkgreen},
    stringstyle=\color{mauve},
    breaklines=true,
    breakatwhitespace=true,
    tabsize=4,
    captionpos=b
}

\def\name{Taylor Thomas }

%% The following metadata will show up in the PDF properties
\hypersetup{
  colorlinks = false,
  urlcolor = black,
  pdfauthor = {\name},
  pdfkeywords = {},
  pdftitle = {},
  pdfsubject = {},
  pdfpagemode = UseNone
}

\parindent = 0.0 in
\parskip = 0.1 in

\begin{document}

\begin{titlepage}
\title{Project 1}
\author{CS444 - Spring 2017 \\ Taylor Thomas, Aravind Parasurama, Justin Brown}
\maketitle
\begin{abstract}
The following document contains information about the producer-consumer problem
solution as well as the steps taken to setup the kernel for the first part of
the assignment. Included is a version control log to show when the work was done
through github.  
\end{abstract}

\thispagestyle{empty} % gets rid of the "0" page number.

\end{titlepage}
%\newpage

\tableofcontents

\newpage

\section{Command Log}
\begin{enumerate}
	\item ssh os-class
	\item cd/scratch/spring2017/13-06
	\item git clone git://git.yoctoproject.org/linux-yocto-3.14
	\item cd linux-yocto-3.14
	\item git checkout v3.14.26
	\item source /scratch/opt/environment-setup-i586-poky-linux.csh
	\item cp /scratch/spring2017/files/config-3.14.26-yocto-qemu.config
	\item make menuconfig
	\item /LOCALVERSION
	\item 1
	\item edit value to: -13-06-hw1 
	\item make -j4 all
	\item cd ..
	\item gdb
	\item in terminal 2: 
	source /scratch/opt/environment-setup-i586-poky-linux.csh
	\item cp /scratch/spring2017/files/bzImage-qemux86.bin.
	\item /scratch/spring2017/files/core-image- lsb-sdk-qemux86.ext3 .
	\item qemu-system- i386 -gdb tcp::5601 -S -nographic -kernel bzImage-qemux86.bin -drive file=core-image- lsb-sdk- qemux86.ext3,if=virtio -enable-kvm -net none-usb-localtime--no-reboot -- append	"root=/dev/vda rw console=ttyS0 debug"
	\item in terminal 1: target remote :5601
	\item continue
	\item for username type: root
	\item uname -a
	\item reboot
	\item qemu-system- i386 -gdb tcp::5601 -S -nographic -kernel linux-yocto- 3.14/arch/x86/boot/bzImage  -drive file=core-image- lsb-sdk- qemux86.ext3,if=virtio -enable- kvm -net none -usb -localtime -- no-reboot -- append	"root=/dev/vda rw console=ttyS0 debug"
	\item target remote :5601
	\item for username type: root
	\item uname -a
	\item reboot
	\item q
	
\end{enumerate}
\section{Concurrency Solution}
\subsection{What do you think the main point of the assignment was?}
The main point of the assignment was to get more experience with thinking about
concurrency, and thinking in the parallel programming paradigm when designing solutions.  
\subsection{How did you personally approach the problem?}
At first we did research to get some background information on the producer
consumer problem. Once we had a good background of the problem we were able to
get started implementing the solution. As far as the implementation went the main
components consisted of the producer function, the consumer function and the
buffer. We began to work on implementing these components one by one until the
program converged in the end. 
\subsection{How did you ensue your solution was correct?}
We ensued that our solution was correct by printing the values that were
produced by the producer and consumed by the consumer. 
\subsection{What did you learn?}
As a group we learned how to use pthreads at a higher level than before. This
assignment also gave me a better understanding of parallel programming. 
\section{Qemu Command-Line Flags}
\begin{itemize}
	\item -gdb tcp::5601 - Open a GDB server on TCP port 5601
	\item -S - Do not start the CPU at startup.
	\item -nographic - Disables graphical output so that QEMU is a command line operation.
	\item -kernel linux-yocto-4.14/arch/x86/boot/bzImage - Uses this bzImage as a kernel image.
	\item -drive file=core-image-quemux86.ext3 - Defines a new drive with the file option set.
	\item if=virtio
	\item -enable-kvm - Enables KVM virtualization report.
	\item -net none - Indicates that no network devices should be configured.
	\item -usb - Enables the USB driver.
	\item -localtime - Gives the local time.
	\item -- no-reboot
	\item -- apend "root=/dev/vda rw console=ttyS0 debug" - Uses the argument as the kernel command line.
\end{itemize}
\section{Version Control Log}
\begin{center}
	\begin{tabular}{| p{0.3\linewidth} | p{0.3\linewidth} | p{0.3\linewidth} |}
		\hline User & Commit Message & Date\\
		\hline bitschift & added producer-consumer& April 20th\\
		\hline bitschift & added some function prototypes& April 20th\\
		\hline bitschift & 	Added rand num and isx86 functionsr& April 21st\\
		\hline bitschift & 	Added 64bit rand num support& April 21st\\
		\hline bitschift & fleshed out producer& April 21st\\
		\hline bitschift & fleshed out consumer& April 21st\\
		\hline bitschift & fixing bugs& April 21st\\
		\hline bitschift & fix bugs& April 21st\\
		\hline bitschift & more bug fixes& April 21st\\
		\hline itztt23 & adding assignment 1 docs& April 21st\\
		\hline itztt23 &added commands for the writeup& April 21st\\
		\hline itztt23 & added qemu command-line flag descriptions& April 21st\\
		\hline &&\\		
	\end{tabular}
\end{center}
\section{Work Log}
\begin{itemize}
	\item April 16th - began working on the kernel part of the assignment
	\item April 20th - began working on the producer function
	\item April 20th - began working on the consumer function
	\item April 20th - built the buffer
	\item April 21st - began writing up the documents
	\item April 21st - solved the problem of the consumer consuming empty values
	\item April 21st - finished the write-up
	\item April 21st - finished the concurrency assignment
\end{itemize}

\end{document}
